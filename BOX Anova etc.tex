\begin{tcolorbox}[colback=Black!5!lightgray,colframe=black!75!black,coltitle=white,title= { analysis of variance,  linear regression and t-test}]
Mathematically, the t-test is equivalent to two other methods: analysis of variance and linear regression\index{analysis!linear regression}. When we have just two groups, all of these methods achieve the same thing: they divide up the variance in the data into variance associated with group identity, and other (residual) variance, and provide a statistic that reflects the ratio between these two sources of variance. This is illustrated with the real data analysed in Table \ref{tab:ancova-compare}. The upper analysis gives results that are equivalent to the t-test in Table \ref{tab:ttestoutcomes}: if you square the t-value it is the same as the F-test. In this case, the p-value from the t-test is half the value of that in the ANOVA: this is because we specified a one-tailed test for the t-test. The p-value would be identical to that from ANOVA if a two-tailed t-test had been used.\\
See \href{http://deevybee.blogspot.com/2017/11/anova-t-tests-and-regression-different.html}{for more details see this blogpost: http://deevybee.blogspot.com/2017/11/anova-t-tests-and-regression-different.html}.
\end{tcolorbox}