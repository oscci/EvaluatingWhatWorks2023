\begin{tcolorbox}[enhanced, breakable,colback=Black!5!lightgray,colframe=black!75!black,coltitle=white,title=How scaled scores are derived]
A scaled score gives an indication of how a given raw score compares to a standardization sample. For instance, if there are 80 6-year-olds in the standardization sample, and their mean raw score is 44, with a standardization of 5, then we can see that a raw score of 39 would be one standard deviation below the mean, and a raw score of 34 would be two standard deviations below the mean.\\
Now suppose we also have normative data on 9-year-olds, and their mean raw score is 46, with standard deviation of 4. In this case, a raw score of 39 is 1.75 standard deviation below the mean, and a raw score of 34 is three standard deviations below the mean. So the same raw score represents a more serious deficit in a 9-year-old than in a 6-year-old, because the typical (normative) score of 9-year-olds is higher than that of 6-year-olds.\\
Converting between raw\index{data!raw} and scaled scores becomes second nature to those who do a lot of psychometric testing, and, as can be seen from the example above, it allows one to compare results on different measures or different age groups on a common metric. But it can be confusing, not least because there are various ways of representing scaled scores.\\
The most basic way of representing a scaled score is as a z-score. This just represents distance from the mean in standard deviation (SD) units. So in the example above, in 6-year-olds, the raw score of 39 would convert to a z-score of -1.0, and a raw score of 34 to a z-score of -2.0. Basically, to get a z-score, you simply subtract the mean for that age group from the obtained score and divide by the SD.\\
Z-scores, however, have some unfortunate characteristics: they involve both positive and negative numbers, and fractional units. When entering data or writing reports, it is much easier to deal with whole numbers. So most psychometric tests simply transform the z-score onto a scale with a more convenient mean and SD. The best-known example is the IQ test, which usually gives a scaled score based on a mean of 100 and standard deviation of 15. So if someone scores one SD below the mean, this is represented as a scaled score of 85; if two SD below the mean, then the scaled score is 70.\\
As if this was not complicated enough, the scaled score can also be directly translated into a percentile (or centile) score, which corresponds to the percentage of people in the population who are expected to obtain a score at least as high as this. This conversion is usually based on the normal distribution, where there is an orderly relationship between percentiles and the z-score. As shown in Figure \@ref(fig:ZDensityCDF), approximately 16\% of people are expected to score 1 SD below the mean, and 3\% score 2 SD below the mean.


%\begin{figure}
\includegraphics[width=0.75\linewidth]{images_bw/zDensityCDF} \captionof{figure}{Distribution of z-scores, showing proportions of people obtaining scores below -1, between -1 and 1, and above 1.}\label{zDensityCDF}
%\end{figure}
  
  
You may wonder why bother with scaled scores, given that percentiles have a more intuitive interpretation. Percentiles are good for communicating how exceptional a given raw score is, but they are not so good for doing statistical analyses. For those interested in understanding why, \href{http://deevybee.blogspot.com/2011/04/short-nerdy-post-about-use-of.html}{this blogpost} gives a short explanation.
\end{tcolorbox}